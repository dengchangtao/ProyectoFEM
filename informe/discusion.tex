\subsection{Bah\'ia rectangular de ancho unitario} 
  
  En primer lugar se verifica lo mencionado en el \'ultimo p\'arrfo de la secci\'on \ref{subsec:ecuaciones}, dado que el menor de los valores propios calculados siempre es igual a 0. Sin embargo, el m\'etodo num\'erico para obtener el vector propio asociado, implementado en la funci\'on \verb;numpy.linalg.eig;, falla, entregando un vector que no se asemeja al exacto. Tambi\'en se verifica la multiplicidad de valores propios con vectores propios diferentes, como los de la figura \ref{fig:multiplicidad}, y que para este caso en particular obedece a las caracter\'isticas de simetr\'i de la bah\'ia, por ejemplo, pero que tambi\'en se pueden encontrar en otros casos, como cuando $\sqrt{n^2+m^2}$ es un n\'umero natural (cuadrado perfecto). Sin embargo, todos los valores propios obtenidos num\'ericamente dieron valores reales, y no complejos, con lo cual se verifica lo expresado en \cite{nica2011}.
  
  En la comparaci\'on entre la soluci\'on anal\'itica y la aproximaci\'on de Elementos Finitos se oberva que cualitativamente los vectores asociados a cada modo propio muestran formas similares, en t\'erminos de los nodos, antinodos y estructura general, lo cual se puede verificar, por ejemplo, en la figura \ref{fig:modos_numericovsanalitico}. En t\'erminos m\'as cuantitativos, la convergencia de los vectores y valores propios asociados a los primeros modos de oscilaci\'on puede tener orden 2 o superior, al medirlo experimentalmente usando la norma de $\mathcal{L}^2$ y de $\mathbb{R}$, como se ve en las \hilight{tablas X e Y}.