  Para el caso de la bah\'i rectangular se verifica lo mencionado en el \'ultimo p\'arrfo de la secci\'on \ref{subsec:ecuaciones}, dado que el menor de los valores propios calculados, en particular, siempre es igual a 0. Sin embargo, el m\'etodo num\'erico para obtener el vector propio asociado, implementado en la funci\'on \verb;numpy.linalg.eig;, falla, entregando un vector que no se asemeja al exacto. Tambi\'en se verifica la multiplicidad de valores propios con vectores propios diferentes, como los de la figura \ref{fig:multiplicidad}, o como se puede ver en la figura \hilight{2}, y que para este caso en particular obedece a las caracter\'isticas de simetr\'i de la bah\'ia, por ejemplo, pero que tambi\'en se pueden encontrar en otros casos como cuando $\sqrt{n^2+m^2}$ es un n\'umero natural (cuadrado perfecto). Sin embargo, todos los valores propios obtenidos num\'ericamente dieron valores no complejos, con lo cual se verifica lo expresado en la secci\'on \ref{subsec:ecuaciones}\cite{nica2011}. Adem\'as, en la figura \hilight{2} se observa que una mejor discretizaci\'on, adem\'as de permitir una mejor representaci\'on de los modos con longitud de onda largas, permite a\~nadir longitudes m\'as cortas a la colecci\'on de modos propios obtenidos por el m\'etodo.
  
  En la comparaci\'on entre la soluci\'on anal\'itica y la aproximaci\'on de Elementos Finitos se oberva que cualitativamente los vectores asociados a cada modo propio muestran notable similitud en t\'erminos de los nodos, antinodos y geometr\'ia general, lo cual se puede verificar, por ejemplo, en la figura \ref{fig:modos_numericovsanalitico}. En t\'erminos cuantitativos, la convergencia de los vectores y valores propios asociados a los primeros modos de oscilaci\'on puede tener orden 2 o superior, al medirlo experimentalmente usando la norma de $\mathcal{L}^2$ y de $\mathbb{R}$ respectivamente, como se ve en las \hilight{tablas X e Y}.
  
%   \\
  
  En el caso de la bah\'ia de Concepci\'on, se verifica que los vectores propios obtenidos, mostrados en la figura \ref{fig:modos_talcahuano}, presentan forma similar a los que se encuentran en la literatura para estudios similares en puertos. Adem\'as se puede observar que los primeros vectores propios son, a grandes rasgos, similares a los de la bah\'ia rectangular, y caracterizados por tener el primero una linea nodal ($u=0$) en la direcci\'on $x$ y dos antinodos en los extremos Norte y Sur de la bah\'ia; el segundo una l\'inea nodal vertical; el tercero dos lineas nodales de orientaci\'on prominentemente horizontal; y el cuarto dos lineas nodales con orientaci\'on NorOeste - SurEste, pero que ya muestran una distorsi\'on m\'as dr\'astica que en los otros casos. 
  
  En particular, los resultados aqu\'i obtenidos par la distribuci\'on espacial de los vectores asociados a los modos propios, presentan, en forma cualitativa, similitud respecto a lo reportado en la literatura \cite{Belloti2012}, lo cual sugiere la validez de la implementaci\'on aqu\'i realizada. Adem\'as, al examinar la informaci\'on disponible en la literatura respecto al tsunami del 2010 \cite{Yamazaki2011}, se encuentra que el per\'iodo principal de oscilacion de \'este es del orden de $40$ minutos, y que corresponde al primer per\'iodo de oscilaci\'on aqu\'i obtenido. Una siguiente etapa de este trabajo incluir\'a la investigaci\'on de los fen\'omenos de resonancia que se pudieron haber observado en la bah\'i de Concepci\'on para el caso del tsunami del 2010.
  
  