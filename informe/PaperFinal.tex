\documentclass[12pt, letterpaper]{article}

\usepackage{graphicx} % Graphics
\usepackage{amsmath, amsfonts, amssymb, amsthm} % Math stuff
\usepackage{enumerate, array, multirow}

\usepackage{caption}
\usepackage{subcaption}
%%%  Language Settings  %%%
%\usepackage[spanish]{babel}  % Spanish language
% \usepackage[applemac]{inputenc}  % Allows for accents!! Useful for spanish documents
%\usepackage[latin1]{inputenc}


%%%  Page settings  %%%
%\usepackage{fullpage} % quick-and-dirty way to get 1in for all margins
\usepackage[left=2cm,top=2cm,right=2cm,bottom=2.5cm,nohead]{geometry}

\title{T\'itulo del proyecto final \'o nombre del art\'iculo}
\author{Jos\'e Galaz, Jaime Soto}
\date{}

\begin{document}
\maketitle

\begin{abstract}
    A menudo la propagaci\'on y superposici\'on de ondas en bah\'ias puede generar oscilaciones de per\'iodos largos que ocasionan inundaci\'on y da\~no inesperado en embarcaciones y estructuras, sin embargo, a partir de las ecuaciones de Navier-Stokes es posible deducir que, cuando estas ondas largas poseen pequeña amplitud, las frecuencias y modos de oscilación corresponden a los valores y vectores propios dados por la ecuación de Helmholtz. En este trabajo ha sido posible desarrollar una formulación variacional del problema, e implementar computacionalmente el método de Galerkin para calcular los modos de resonancia aproximados para una bah\'ia de geometría arbitraria. Ha sido posible validar con soluciones anal\'iticas y se ha aplicado el método para estudiar la bah\'ia de Concepción, Chile, encontrando que los períodos de oscilación son de orden de magnitud esperado y la forma de los modos principales concuerda con lo que se reporta en la literatura \cite{Belloti2012}.
\end{abstract}

\section{Introducci\'on}

El fen\'omeno de resonancia al interior de una regi\'on  semicerrada reviste una especial importancia en el estudio del comportamiento de ondas que se propagan desde el oc\'eano hacia zonas costeras y que pueden inducir la amplificaci\'on de estas al ingresar a una regi\'on semicerrada \cite{Kowalik1993}. El fen\'omeno de resonancia es de importancia en el diseño de puertos y comportamiento de naves al interior de estos \cite{Diaz2006web}, la inundaci\'on por efecto de tormentas y marejadas \cite{Kowalik1993} y en la amplificaci\'on y/o aparici\'on de ondas de tsunami tard\'ia, horas despu\'es de la llegada de la primera onda \cite{Kowalik1993}. Esta \'ultima podría ser una explicaci\'on para describir las caracter\'isticas que tuvo el tsunami de Maule 2010 \cite{Cyper2012web}.

Existen varias formas de aproximarse a una identificaci\'on y cuantificaci\'on de los modos propios de oscilaci\'on al interior de una bah\'ia. Una aproximaci\'on es propagar varias ondas (distinta frecuencia y direcci\'on) hacia el interior y cuantificar la amplificaci\'on de estas. Esta propagaci\'on puede ser lineal o no lineal \cite{Mei2005}. Otra aproximaci\'on consiste en calcular directamente los modos propios de oscilaci\'on de la ba\'hia \cite{Belloti2012, Mei2005}. Esta \'ultima aproxiamci\'on es la utilizada en este trabajo.

El objetivo de este trabajo es determinar los modos de oscilaci\'on de una bah\'ia utilizando una aproximaci\'on por elementos finitos. La mayor dificultad que presenta una aproximaci\'on directa para la determinaci\'on de los modos propios es la seleccionar una condici\'on de borde apropiada a cada caso \cite{Mei2005, Rabino2009}. Como una primera aproxiamci\'on se considerar\'a una bah\'ia cerrada.

Para encontrar una aproximaci\'on a los modos propios se resuelve la ecuaci\'on de Helmholtz, mediante el m\'etodo de elementos finitos, aplicado a una bah\'ia cerrada con condiciones de borde Neumman $u,_i  n_i = 0$. Las ecuaci\'on de Helmholtz es derivada de la ecuacion de Navier-Stokes y presentada en su formualci\'on fuerte. Se obtiene la formulaci\'on variacional y galerkin asociada al problema fuerte y se implementan elementos tipo tri3 para la resolucion de la ecuacion matricial. El modelo es validado en una geometr\'ia rectangular cuya soluci\'on anal\'itica es conocida y alicado al caso real de la Bah\'ia de Talcahuano


\section{Marco Te\'orico}
  \subsection{Ecuaciones fundamentales}
  Sea  $\Omega' \subset \mathbb{R}^3$ el dominio de inter\'es, $t_f>0$ el tiempo final de la simulaci\'on,   $\vec v : [0,t_f]\times \Omega' \rightarrow  \mathbb{R}^3$ el campo de velocidad, y $p : [0,t_f] \times \Omega' \rightarrow \mathbb{R}$ el campo de presiones, de un fluido incompresible con densidad $\rho \in \mathbb{R}^+$, que escurre sobre un fondo (topograf\'ia - batimetr\'ia) de forma dada por $b:\Omega' \rightarrow \mathbb{R}$. Si se desprecian efectos disipativos y se consideran fuerzas de volumen $\vec f_b = (0,0,-g)^T$, con $g$ la aceleraci\'on de gravedad, las ecuaciones de Navier-Stokes son \cite{toro}
\begin{align}
  \begin{split}
    \nabla \cdot \vec v &= 0 \\
    \frac{\partial }{\partial t}\vec v + \nabla \cdot \vec v \otimes \vec v  &= -\frac{1}{\rho}\nabla p + \vec f_b    \\
    v_3|_{z=\eta	} &= \frac{\partial \eta}{\partial t}+(\vec v \cdot \nabla )\eta \\
    v_3|_{z=b} &= \frac{\partial b}{\partial t} + (\vec v \cdot \nabla )b \\    
    p|_{z = \eta} &= 0
  \end{split}  
  \label{NS-incompresible}
\end{align}

En particular, en una bah\'ia de interior $\Omega \subset \mathbb{R}^2$, con borde impermeable $\partial\Omega_h$ y linea nodal $\partial \Omega_g$ tales que $\partial \Omega = \partial \Omega_h \cup \partial \Omega_g$ , y bajo el supuesto que las ondas son suficientemente largas, de forma que las aceleraciones verticales no tienen influencia significtiva sobre el perfil de presiones hidrost\'atico, y por medio de integraci\'on vertical entre el fondo $b$ y la superficie libre $\eta$, las Ecuaciones No Lineales de Aguas Someras (Non-Linear Shallow water Equations, NSWE) son, 

\begin{align}  \begin{split}
\frac{\partial}{\partial t}\left(\eta\right)+\frac{\partial}{\partial x}\left(hu\right)+\frac{\partial}{\partial y}\left(hv\right) & =0  \text{       si } x \in \Omega\\
  \frac{\partial}{\partial t}\left(hu\right)+\frac{\partial}{\partial x}(hu^{2}+\frac{1}{2}gh^{2})+\frac{\partial}{\partial y}(huv) & =-gh\frac{\partial}{\partial x}b  \text{   si } x\in\Omega\\
  \frac{\partial}{\partial t}\left(hu\right)+\frac{\partial}{\partial x}(huv)+\frac{\partial}{\partial y}(hv^{2}+\frac{1}{2}gh^{2}) & =-gh\frac{\partial}{\partial y}b  \text{     si } x \in \Omega\\
  (hu,hv) \cdot \vec n &= 0  \text{ si    } x\in\partial \Omega_h \\
   \eta &= 0  \text{ si    } x \in \partial \Omega_g  
  \end{split}
  \label{eq:nswe_cart}
  \end{align}
donde, viendo la figura \ref{fig:vars}, $h:(t,x,y) \in [0,t_f] \times \Omega \to \eta(t,x,y)-b(x,y) \in \mathbb{R}^+$ es la altura de la columna de agua,  y $(u,v)$ las componentes de velocidad horizontal promediadas en la vertical, dadas por
 $$
  (u,v)=\frac{1}{h}\int_{b}^\eta (v_1,v_2)dz
 $$
  
  \begin{figure}
    \centering
    \includegraphics[width=10cm]{figs/variables.pdf}    
    \caption{ Vista esquem\'tica de ls varibles hidrodin\'amicas definidas para las Ecuaciones No Lineales de Aguas Someras (NSWE).}
    \label{fig:vars}
  \end{figure}

Las ecuaciones en \eqref{eq:nswe_cart}, forman un sistema hiperb\'olico de ecuaciones diferenciales parciales no lineales, que admite ondas de choque e interfaces seco-mojado, cuando $h$ tiende a $0$. Sin embargo, es posible obtener una linealizaci\'on del sistema \eqref{eq:nswe_cart} si se consideran peque\~nas perturbaciones a una masa de agua en reposo, es decir si
$$
	\eta=h_0+\eta' \hspace{.5cm} u=0+u',\hspace{.5cm} v = 0 + v'
$$
donde la notaci\'on $\square'$ indica peque\~nas perturbaciones sobre $\square$, y $h_0:\Omega\rightarrow \mathbb{R}$ representa la altura de la columna de agua, de forma que  \footnote{Aqu\'i el lector debe notar que este supuesto es equivalente a asumir que las ondas son de amplitud peque\~na.}
$$\left|\frac{\eta'(t,x,y)}{h_0(x,y)}\right|<<1$$
para cualquier $(t,x,y) \in [0,t_f]\times\Omega$. Bajo estos supuestos, se deducen las Ecuaciones Lineales de Aguas Someras (LSWE), dadas por
\begin{align}
	\begin{split}
	\frac{\partial \eta'}{\partial t}+\frac{\partial h_0u'}{\partial x}+\frac{\partial h_0v'}{\partial y} = 0  \text{ si } x \in \Omega\\
    \frac{\partial u'}{\partial t} + g\frac{\partial \eta'}{\partial x}=0  \text{ si } x \in \Omega\\
    \frac{\partial v'}{\partial t} + g\frac{\partial \eta'}{\partial y} = 0  \text{ si } x \in \Omega \\
    (h_0u',h_0v') \cdot \vec n = 0 & \text{ si } x\in\partial \Omega_h \\
   \eta' = 0  \text{ si } x \in \partial \Omega_g  
    \end{split}
    \label{swe}
\end{align}

    Si ademas $\eta',u',v' \in \mathcal{C}^2(\Omega)$, multiplicando la segunda y tercera ecuaci\'on de \eqref{swe}, por $h$ y derivando respecto a $x$ e $y$ es cierto que 
    \begin{equation}
    	\begin{split}
    	\frac{\partial^2 \eta}{\partial t^2} - \nabla \cdot( gh \nabla \eta) = 0, \text{ si } x \in \Omega \\
        \frac{\partial \eta}{\partial \vec n} = 0 \text{ si } x \in \partial \Omega_h \\
        \eta = 0 \text{ si }x \in \partial \Omega_g
        \end{split}
     \label{eqonda}
    \end{equation}

    donde se cambi\'o de notaci\'on al usar $\square$ como $\square'$. Las ecuaciones \eqref{eqonda} corresponden a la ecuaci\'on lineal de onda,  de celeridad $c=\sqrt{gh}$, revelando la similitud entre la propagaci\'on de ondas de amplitud peque\~na en una bah\'ia, con la de ondas ac\'usticas o el\'asticas lineales.
    
    Finalmente, si se estudian ondas estacionarias, es posible separar variables y escribir (abusando de notaci\'on en $u$), con $u:\Omega \rightarrow \mathbb{R}$
    \begin{equation}
    	\eta(t,x,y) =Re\left\{ u(x,y) e^{-i \omega t}\right\}
    \end{equation}
    
    lo cual, sustituyendo en \eqref{eqonda}, conduce a 
    
    \begin{equation}
      \begin{split}
    	\nabla \cdot \left( gh_0 \nabla u\right) + \omega^2 u = 0 \text{ si } x \in \Omega\\
        \frac{\partial u}{\partial \vec n} = 0 \text{ si } x \in \partial \Omega_h \\
        u = 0 \text{ si } x \in \partial \Omega_g
      \end{split}
        \label{helmholtz}
    \end{equation}
    
    El sistema \eqref{helmholtz} es m\'as conocido como la Ecuaci\'on de Helmholtz, y denota un problema de valores y vectores propios del operador diferencial de la ecuaci\'on de Poisson ($\nabla \cdot c^2 \nabla \square$). Es posible demostrar \cite{nica2011}, que la extensi\'on d\'ebil del operador de Poisson definido sobre $L^2(\Omega,\mathbb{R})$ posee una cantidad numerable de valores y vectores propios $(\lambda_n, u_n)_{n\in\mathbb{N}}$, tales que $u_n\neq 0$ y $0\leq \lambda_1$ y $\lambda_n \leq \lambda_{n+1}$. Adem\'as es f\'acil verificar, que para el caso en que $ \partial \Omega _g = \varnothing$, como es el caso de una bah\'ia cerrada, si $u(x,y)=K \in \mathbb{R}$ para cualquier $(x,y)\in\Omega$, entonces \eqref{helmholtz} se satisface trivialmente y $\lambda=0$ es el primer valor propio, lo cual corresponde f\'isicamente a cuando la superficie libre del agua est\'a a un nivel constante, es decir, cuando la masa de agua est\'a en reposo. Lo anterior se debe tener en consideraci\'on al examinar los valores propios obtenidos num\'ericamente.
    %f\'isicamente, como $T=2\pi/\sqrt{\lambda}=\infty$ y $u=K$, indica que el primer modo de vibraci\'on de la bah\'ia corresponde uniformes de la superficie libre $\eta$, en todo el dominio.
%     
%     @article{nica,
% 	title="Eigenvalues and Eigenfunctions of the Laplacian",
% 	author="Mihai Nica",
% 	journal="The Waterloo Mathematics Review",
% 	year="2011",
% 	volume=1,
% 	number=2,
% 	pages={23--34},
% }
    
    
  \subsection{Formulaci\'on variacional y de Galerkin}
  \subsection{Formulacion variacional y ecuaci\'on matricial}
La formulación fuerte del problema de valor de frontera es: (ecuaci\'on \eqref{helmholtz}

\begin{align*}
\text{Dados $g$, $h: \Omega \rightarrow \mathbb{R}$, encontrar $(u, \omega^2)$ tal que:}\\
\omega^2 u + (gh u,_i),_i  = 0, \ \ \ \ & \boldsymbol{x} \in \Omega \\
\frac{\partial u}{\partial n} = u,_i n_i = 0, \ \ \ \ & \boldsymbol{x} \in \partial \Omega_h \ & (S) \\
u=0, \ \ \ \ &\boldsymbol{x} \in \partial \Omega_g\\
\end{align*}

Sean:\\ \\
$ \mathcal{V} = \left \{ v \in H^1 (\Omega, \mathbb{R}) \ /\  v|_{\partial \Omega_g} = 0 \right \}$\\
$ \mathcal{S} = \left \{ u \in H^1 (\Omega) \ /\  u,_i n_i = 0, \ \ \boldsymbol{x} \in \partial \Omega_h \right \}$\\ 

multiplicando la ecuaci\'on de Helmholtz por $-v \in \mathcal{V}$ e integrando por partes:\\

$$-\int_{\Omega} v \left( \omega^2 u + (gh u,_i),_i \right) \mathrm{d}\boldsymbol{x} = 0$$

$$-\int_{\Omega} v ( \omega^2 u )\mathrm{d}\boldsymbol{x} -\int_{\partial \Omega} v (gh u,_i) n_i \mathrm{d} S
+\int_{\Omega} v,_i g h u,_i \mathrm{d}\boldsymbol{x} = 0$$

Pero, por condici\'on de borde: $u,_i n_i = 0 \ \forall \boldsymbol{x} \in \partial \Omega_h$, luego

$$-\int_{\Omega} v ( \omega^2 u )\mathrm{d}\boldsymbol{x} 
+\int_{\Omega} v,_i g h u,_i \mathrm{d}\boldsymbol{x} = 0$$

O, en forma abstracta:
\begin{equation}
a(v, u) - \omega^2 (v, u) = 0
\label{eq:debil_abstracta}
\end{equation}

Luego, 

\begin{align*}
\text{Dados $g$, $h: \Omega \rightarrow \mathbb{R}$, encontrar $(u, \omega^2)$, $u \in \mathcal{S} $, $\omega \in \mathbb{R}$, tal que:}\\
a(v, u) - \omega^2 (v, u) = 0 \ \ \ \ \  & (W) \\
\end{align*}

Formulaci\'on Galerkin:

Sean $u^h \in \mathcal{S}^h \left( \equiv \mathcal{V}^h \right)$ con $\mathcal{V}^h \in \mathcal{V}$

entonces:

$$ a(v^h, u^h) - \omega^2 (v^h, u^h) = 0 $$

Luego la formulaci\'on galerkin queda:

\begin{align*}
\text{Dados $g$, $h: \Omega \rightarrow \mathbb{R}$, encontrar $(u, \omega^2)$, $u \in \mathcal{V}^h $, $\omega \in \mathbb{R}$, tal que:}\\
a(v^h, u^h) - \omega^2 (v^h, u^h) = 0 \ \ \ \ \  & (W) \\
\end{align*}

Para hallar las ecuaciones matriciales asociadas, debemos expresar $v^h$ y $u^h$ en t\'erminos de las funciones de forma $N(\boldsymbol{x})$:

$$v^h = \sum_{A \in \eta}N_A(\boldsymbol{x})v_A$$
$$u^h = \sum_{A \in \eta}N_A(\boldsymbol{x})u_A$$

donde 
$\eta$: Nodos del dominio\\

Reemplazando, se tiene:

\begin{equation*}
\begin{split}
\sum_{A \in \eta} v_A 
\left \{
\sum_{B \in \eta}(N_A(\boldsymbol{x}), N_B(\boldsymbol{x})) u_B - 
\omega^2 \sum_{B \in \eta} a(N_A(\boldsymbol{x}), N_B(\boldsymbol{x})) u_B 
\right \} = 0
\end{split}
\end{equation*}

Como $v_A \neq 0$, entonces:
$$\left( \sum_{B \in \eta}(N_A(\boldsymbol{x}), N_B(\boldsymbol{x})) - 
\omega^2 \sum_{B \in \eta} a(N_A(\boldsymbol{x}), N_B(\boldsymbol{x}))\right) u_B =0 $$

Luego, el problema matricial queda:

$$(M - \omega^2 K)\boldsymbol{u_B} = 0$$

donde:\\

$K_{AB} = a(N_A(\boldsymbol{x}), N_B(\boldsymbol{x})) = \int_{\Omega} gh N_A,_i(\boldsymbol{x}) N_B,_j(\boldsymbol{x}) \mathrm{d}\boldsymbol{x} $\\

$M_{AB} = (N_A(\boldsymbol{x}), N_B(\boldsymbol{x})) = \int_{\Omega} N_A(\boldsymbol{x}) N_B(\boldsymbol{x}) \mathrm{d}\boldsymbol{x} $\\

A nivel de elemento:\\

$K_{ab} = a(N_a(\boldsymbol{x}), N_b(\boldsymbol{x})) = \int_{\Omega_e} gh N_a,_i(\boldsymbol{x}) N_b,_j(\boldsymbol{x}) \mathrm{d}\boldsymbol{x} $\\

$M_{ab} = (N_a(\boldsymbol{x}), N_b(\boldsymbol{x})) = \int_{\Omega_e} N_a(\boldsymbol{x}) N_b(\boldsymbol{x}) \mathrm{d}\boldsymbol{x} $\\

El tipo de elemento particular utilizado se explica en la secci\'on de implementaci\'on







\section{Resultados}
  asdfsafa
validajilasdfjsdijfaosjo
asdfasdf

asdfkalskfjczxv,zmnvc

  \subsection{Aplicaci\'on a la bah\'ia de Concepci\'on}

A partir de informaci\'on  de cartas n\'auticas del Servicio Hidrogr\'afico y Oce\'anico de la Armada de Chile \cite{shoa} se interpol\'o la topograf\'ia y batimetr\'ia correspondiente a la zona de la bah\'ia de Concepci\'on, y se utiliz\'o el software computacional AnuGA \cite{anuga} para generar la malla de elementos triangulares en el lugar de inter\'es, como se puede ver en la figura \ref{fig:bati-talcahuano}. La malla posee un total de 2697 nodos y 57600 tri\'angulos, y se impuso que el \'area de cada triangulo fuera menor que $57600m^2$. 

La condici\'on de borde por el lado que da hacia el oc\'eano no es trivial de implementar, y como una primera aproximaci\'on se simul\'o como un borde s\'olido, lo cual es consistente con el caso l\'imite en que las ondas quedan completamente atrapadas al interior. De esta forma, la bah\'ia completa se consider\'o como si estuviera cerrada por todos los bordes. 

Dadas estas definiciones, los seis primeros modos de oscilaci\'on y los per\'iodos asociados se encuentran representados en la figura \ref{fig:modos_talcahuano}.

\begin{figure}
  \centering
  \includegraphics[width=15cm]{figuras/04bati+malla.png}
  \caption{ Batimetr\'ia y topograf\'ia de Talcahuano y malla triangular utilizada en la bah\'ia de Concepci\'on}  
  \label{fig:bati-talcahuano}
\end{figure}

\begin{figure}
  \begin{subfigure}{0.5\textwidth}
    \includegraphics[width=\textwidth]{figuras/modos1.png}
  \end{subfigure}
  ~
  \begin{subfigure}{0.5\textwidth}
    \includegraphics[width=\textwidth]{figuras/modos2.png}
  \end{subfigure}
  ~
  \begin{subfigure}{0.5\textwidth}
    \includegraphics[width=\textwidth]{figuras/modos3.png}
  \end{subfigure}
  ~
  \begin{subfigure}{0.5\textwidth}
    \includegraphics[width=\textwidth]{figuras/modos4.png}
  \end{subfigure}
  \\
  \begin{subfigure}{0.5\textwidth}
    \includegraphics[width=\textwidth]{figuras/modos5.png}
  \end{subfigure}
  ~
  \begin{subfigure}{0.5\textwidth}
    \includegraphics[width=\textwidth]{figuras/modos6.png}
  \end{subfigure}
  
  \caption{Primeros seis modos y per\'iodos de oscilaci\'on obtenidos para la configuraci\'on de la bah\'ia de Concepci\'on}
  \label{fig:modos_talcahuano}
\end{figure}
  \input{resultados.tex}
En esta secci\'on usted debe presentar los resultados obtenidos de su investigaci\'on, en una secuencia l\'ogica. En esta secci\'on usted NO debe interpretar los resultados ni concluir algo a partir de estos. Sin embargo, si puede destacar algunos resultados importantes (por ejemplo, {\it es importante notar que el valor del estiramiento principal mayor alcanzan valores sobre 1.5, que corresponden a una deformaci\'on axial del 50 \%}) que ser\'an el punto de partida de sus conclusiones en la secci\'on de discusi\'on. Los gr\'aficos deben ser confeccionados con mucha atenci\'on, de manera de ser muy claros y no tener demasiada informaci\'on que dificulte leerlos. Sin embargo, sea elegante y eficiente, y no genere gr\'aficos que no reporten informaci\'on importante. No olvide agregar leyendas para las figuras (con el comando {\tt \textbackslash caption}) que ayuden la compresi\'on de estas (no repita informaci\'on en el texto principal). Para un ejemplo, vea la figura \ref{fig:tvconv}.



\section{Discusion}
En esta secci\'on usted debe interpretar sus resultados presentados en la secci\'on anterior, y compararlos/contrastarlos con resultados conocidos previamente en otros art\'iculos o libros. Lo anterior le permitir\'a generar conclusiones a partir de la interpretaci\'on de sus resultados. Sea claro en los pasos de su razonamiento que lo llevan a sus conclusiones. La secci\'on de discusi\'on debe responder las preguntas o verificar las hip\'otesis planteadas en la secci\'on de {\it introducci\'on}. Es recomendable tambi\'en incluir en la discusi\'on cuales son las limitantes de su investigaci\'on, y c\'omo pueden afectar los resultados y conclusiones obtenidas. Finalmente, es recomendable incluir posibles ideas \'o proyectos futuros que nacen a ra\'iz del trabajo de investigaci\'on presentado en el art\'iculo.


\begin{thebibliography}{10}

\bibitem{nica2011}
  M. Nica. Eigenvalues and Eigenfunctions of the Laplacian. \emph{The Waterloo Mathematics Review} 2011; {\bf 1}, No.2, 23--34.
\bibitem{hughes2000}
  T.J.R. Hughes, \emph{The finite element method: Linear static and dynamic finite element analysis}. Dover Publications, 2000.
  
\bibitem{zienkiewicz2006}
  O.C. Zienkiewicz, R.L. Taylor, J.Z. Zhu, \emph{The finite element method: Its basis and fundamentals}. Sixth edition, Butterworth-Heinemann, 2006.  

\bibitem{riceweb} {\tt http://www.ruf.rice.edu/~bioslabs/tools/report/reportform.html }

\bibitem{goktepekuhl2009}
	S. G\"{o}ktepe and E. Kuhl.Computational modeling of cardiac electrophysiology: A novel finite element approach. \emph{International Journal for Numerical Methods in Engineering} 2009; {\bf 79}, 156--178.
	
\bibitem{Kowalik1993}
	Z. Kowalik and T.S. Murty. \emph{Numerical modeling of ocean dynamics. Advanced Series.on Ocean Engineering, vol. 5. }World Scientific, Singapore. 1993; 
	
\bibitem{Mei2005}
	C. Mei, M. Stiassnie and D. Yue \emph{Theory and applicaction of suface ocean Waves, vol. 5. }World Scientific, Singapore. 2005; 

\bibitem{Diaz2006web} {\tt http://www.tdx.cat/bitstream/handle/10803/10618/}

\bibitem{Cyper2012web} {\tt http://ciperchile.cl/2012/01/18/tsunami-paso-a-paso-los-escandalosos-errores-y-omisiones-del-shoa-y-la-onemi/
}

\bibitem{Belloti2012}
	Belloti et al.Modal analysis of semi-enclosed basins. \emph{Coastal Engineering} 2012; {\bf 64}, 16--25.

\bibitem{Rabino2009}
	A.B. Rabinovich. Seiches and harbor oscillations. \emph{Handbook of coastal and ocean engineering. World Scientific Publ} 2009; {\bf 64}, 193--236.



\end{thebibliography}

\appendix

\section{Sobre los ap\'endices}
	Los ap\'endices contienen informaci\'on que no es de car\'acter esencial para entender el art\'iculo. Muchas derivaciones de ecuaciones, demostraciones de teoremas (con la salvedad de art\'iculos en revistas de matem\'atica), tablas de datos, etc., son inclu\'idas en el ap\'endice.
	
\section{Sobre la evaluaci\'on del art\'iculo final}
	El art\'iculo final ser\'a evaluado tanto en los aspectos acad\'emicos (calidad de los resultados obtenidos) como en aspectos de presentaci\'on y redacci\'on. En particular, se considerar\'a en la evaluaci\'on la redacci\'on, diagramaci\'on, ortograf\'ia, correcta referenciaci\'on (incluir las citas y bibliograf\'ia), gr\'aficos y figuras, entre otros.




\end{document}
