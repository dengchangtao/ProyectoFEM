La formulación fuerte del problema de valor de frontera es: (ecuaci\'on \eqref{helmholtz}

\begin{align*}
\text{Dados $g$, $h: \Omega \rightarrow \mathbb{R}$, encontrar $(u, \omega^2)$ tal que:}\\
\omega^2 u + (gh u,_i),_i  = 0, \ \ \ \ & \boldsymbol{x} \in \Omega \\
\frac{\partial u}{\partial n} = u,_i n_i = 0, \ \ \ \ & \boldsymbol{x} \in \partial \Omega_h \ & (S) \\
u=0, \ \ \ \ &\boldsymbol{x} \in \partial \Omega_g\\
\end{align*}

Sean:\\ \\
$ \mathcal{V} = \left \{ v \in H^1 (\Omega, \mathbb{R}) \ /\  v|_{\partial \Omega_g} = 0 \right \}$\\
$ \mathcal{S} = \left \{ u \in H^1 (\Omega) \ /\  u,_i n_i = 0, \ \ \boldsymbol{x} \in \partial \Omega_h \right \}$\\ 

multiplicando la ecuaci\'on de Helmholtz por $-v \in \mathcal{V}$ e integrando por partes:\\

$$-\int_{\Omega} v \left( \omega^2 u + (gh u,_i),_i \right) \mathrm{d}\boldsymbol{x} = 0$$

$$-\int_{\Omega} v ( \omega^2 u )\mathrm{d}\boldsymbol{x} -\int_{\partial \Omega} v (gh u,_i) n_i \mathrm{d} S
+\int_{\Omega} v,_i g h u,_i \mathrm{d}\boldsymbol{x} = 0$$

Pero, por condici\'on de borde: $u,_i n_i = 0 \ \forall \boldsymbol{x} \in \partial \Omega_h$, luego

$$-\int_{\Omega} v ( \omega^2 u )\mathrm{d}\boldsymbol{x} 
+\int_{\Omega} v,_i g h u,_i \mathrm{d}\boldsymbol{x} = 0$$

O, en forma abstracta:
\begin{equation}
a(v, u) - \omega^2 (v, u) = 0
\label{eq:debil_abstracta}
\end{equation}

Luego, 

\begin{align*}
\text{Dados $g$, $h: \Omega \rightarrow \mathbb{R}$, encontrar $(u, \omega^2)$, $u \in \mathcal{S} $, $\omega \in \mathbb{R}$, tal que:}\\
a(v, u) - \omega^2 (v, u) = 0 \ \ \ \ \  & (W) \\
\end{align*}

Formulaci\'on Galerkin:

Sean $u^h \in \mathcal{S}^h \left( \equiv \mathcal{V}^h \right)$ con $\mathcal{V}^h \in \mathcal{V}$

entonces:

$$ a(v^h, u^h) - \omega^2 (v^h, u^h) = 0 $$

Luego la formulaci\'on galerkin queda:

\begin{align*}
\text{Dados $g$, $h: \Omega \rightarrow \mathbb{R}$, encontrar $(u, \omega^2)$, $u \in \mathcal{V}^h $, $\omega \in \mathbb{R}$, tal que:}\\
a(v^h, u^h) - \omega^2 (v^h, u^h) = 0 \ \ \ \ \  & (W) \\
\end{align*}

Para hallar las ecuaciones matriciales asociadas, debemos expresar $v^h$ y $u^h$ en t\'erminos de las funciones de forma $N(\boldsymbol{x})$:

$$v^h = \sum_{A \in \eta}N_A(\boldsymbol{x})v_A$$
$$u^h = \sum_{A \in \eta}N_A(\boldsymbol{x})u_A$$

donde 
$\eta$: Nodos del dominio\\

Reemplazando, se tiene:

\begin{equation*}
\begin{split}
\sum_{A \in \eta} v_A 
\left \{
\sum_{B \in \eta}(N_A(\boldsymbol{x}), N_B(\boldsymbol{x})) u_B - 
\omega^2 \sum_{B \in \eta} a(N_A(\boldsymbol{x}), N_B(\boldsymbol{x})) u_B 
\right \} = 0
\end{split}
\end{equation*}

Como $v_A \neq 0$, entonces:
$$\left( \sum_{B \in \eta}(N_A(\boldsymbol{x}), N_B(\boldsymbol{x})) - 
\omega^2 \sum_{B \in \eta} a(N_A(\boldsymbol{x}), N_B(\boldsymbol{x}))\right) u_B =0 $$

Luego, el problema matricial queda:

$$(M - \omega^2 K)\boldsymbol{u_B} = 0$$

donde:\\

$K_{AB} = a(N_A(\boldsymbol{x}), N_B(\boldsymbol{x})) = \int_{\Omega} gh N_A,_i(\boldsymbol{x}) N_B,_i(\boldsymbol{x}) \mathrm{d}\boldsymbol{x} $\\

$M_{AB} = (N_A(\boldsymbol{x}), N_B(\boldsymbol{x})) = \int_{\Omega} N_A(\boldsymbol{x}) N_B(\boldsymbol{x}) \mathrm{d}\boldsymbol{x} $\\

A nivel de elemento:\\

$K_{ab} = a(N_a(\boldsymbol{x}), N_b(\boldsymbol{x})) = \int_{\Omega_e} gh N_a,_i(\boldsymbol{x}) N_b,_i(\boldsymbol{x}) \mathrm{d}\boldsymbol{x} $\\

$M_{ab} = (N_a(\boldsymbol{x}), N_b(\boldsymbol{x})) = \int_{\Omega_e} N_a(\boldsymbol{x}) N_b(\boldsymbol{x}) \mathrm{d}\boldsymbol{x} $\\

Los elementos utilizados en este estudio fueron elementos isoparam\'etricos Tri3. 






