La formulaci\'on fuerte del problema de valor de frontera \eqref{helmholtz}, es

\begin{align*}
\text{Dados $g$, $h: \Omega \rightarrow \mathbb{R}$, encontrar $(u, \omega^2)$ tal que:}\\
\omega^2 u + (gh u,_i),_i  = 0, \ \ \ \ & \boldsymbol{x} \in \Omega \\
\frac{\partial u}{\partial n} = u,_i n_i = 0, \ \ \ \ & \boldsymbol{x} \in \partial \Omega_h \ & (S) \\
u=0, \ \ \ \ &\boldsymbol{x} \in \partial \Omega_g\\
\end{align*}

 Sean 
 $ \mathcal{V} = \left \{ v \in H^1 (\Omega, \mathbb{R}) \ /\  v|_{\partial \Omega_g} = 0 \right \}$ y
  $ \mathcal{S} = \left \{ u \in H^1 (\Omega) \ /\  u,_i n_i = 0, \ \ \boldsymbol{x} \in \partial \Omega_h \right \}$ los espacios test y de soluci\'on. Al multiplicar la ecuaci\'on de Helmholtz por $-v \in \mathcal{V}$ e integrar por partes se obtiene
$$-\int_{\Omega} v \left( \omega^2 u + (gh u,_i),_i \right) \mathrm{d}\boldsymbol{x} = 0$$
$$-\int_{\Omega} v ( \omega^2 u )\mathrm{d}\boldsymbol{x} -\int_{\partial \Omega} v (gh u,_i) n_i \mathrm{d} S
+\int_{\Omega} v,_i g h u,_i \mathrm{d}\boldsymbol{x} = 0$$
y por las condiciones de borde Neumann y Dirichlet, se deduce
$$-\int_{\Omega} v ( \omega^2 u )\mathrm{d}\boldsymbol{x} 
+\int_{\Omega} v,_i g h u,_i \mathrm{d}\boldsymbol{x} = 0$$
que definiendo $a(v,u)=\int_{\Omega} v,_i g h u,_i \mathrm{d}\boldsymbol{x}$ y $(v,u)=\int_{\Omega} v u\mathrm{d}\boldsymbol{x} $, se puede escribir como
\begin{equation}
a(v, u) - \omega^2 (v, u) = 0
\label{eq:debil_abstracta}
\end{equation}

Entonces, la formulaci\'on variacional asociada al problema fuerte de la ecuaci\'on de Helmholtz es

\begin{equation}
  \begin{split}
  \text{Encuentre $u\in\mathcal{S}$ y $\omega \in \mathbb{R}$  tales que} \\
    a(v,u) - \omega^2 (v,u) = 0 \\
    \text{ para cualquier $v \in \mathcal{V}$}
  \end{split}
  \tag{W}
\end{equation}


Sean $u^h \in \mathcal{S}^h \left( \equiv \mathcal{V}^h \right)$ con $\mathcal{V}^h \in \mathcal{V}$ y tal que $\mathcal{V}^h$ tiene dimensi\'on finita, entonces debe cumplirse que 

$$ a(v^h, u^h) - \omega^2 (v^h, u^h) = 0 $$

Luego la formulaci\'on galerkin se puede plantear como

\begin{equation}
  \begin{split}
  \text{Encuentre $u^h\in\mathcal{S}^h$ y $\omega^h \in \mathbb{R}$  tales que} \\
    a(v^h,u^h) - (\omega^h)^2 (v^h,u^h) = 0 \\
    \text{ para cualquier $v^h \in \mathcal{V}^h$}
  \end{split}
  \tag{\text{$W^{FEM}$}}
\end{equation}


Para hallar las ecuaciones matriciales asociadas, debemos expresar $v^h$ y $u^h$ en t\'erminos de las funciones de forma $N(\boldsymbol{x})$, es decir, 

$$v^h = \sum_{A \in \eta-\eta_g}N_A(\boldsymbol{x})v_A$$
$$u^h = \sum_{A \in \eta-\eta_g}N_A(\boldsymbol{x})u_A$$

donde 
$\eta$ es el conjunto de \'indices de los  nodos del dominio y $\eta_g$ el conjunto de \'indices asociados a nodos con condiciones de borde de Dirichlet

Reemplazando, se tiene que para todo $ v^h\in \mathcal{V}^h$ debe cumplirse

\begin{equation*}
\begin{split}
\sum_{A \in \eta-\eta_g} v_A 
\left \{
\sum_{B \in \eta-\eta_g}(N_A(\boldsymbol{x}), N_B(\boldsymbol{x})) u_B - 
(\omega^h)^2 \sum_{B \in \eta} a(N_A(\boldsymbol{x}), N_B(\boldsymbol{x})) u_B 
\right \} = 0
\end{split}
\end{equation*}

Como $v_A \neq 0$, entonces es equivalente a que para cualquier $A\in \eta-\eta_g$
$$\left( \sum_{B \in \eta-\eta_g}(N_A(\boldsymbol{x}), N_B(\boldsymbol{x})) - 
(\omega^h)^2 \sum_{B \in \eta-\eta_g} a(N_A(\boldsymbol{x}), N_B(\boldsymbol{x})) u_B\right) =0 $$

que en forma matricial se escribe 

$$(\vec M - (\omega^h)^2 \vec K)\boldsymbol{u_B} = 0$$

donde 

$K_{AB} = a(N_A(\boldsymbol{x}), N_B(\boldsymbol{x})) = \int_{\Omega} gh N_A,_i(\boldsymbol{x}) N_B,_i(\boldsymbol{x}) \mathrm{d}\boldsymbol{x} $\\

$M_{AB} = (N_A(\boldsymbol{x}), N_B(\boldsymbol{x})) = \int_{\Omega} N_A(\boldsymbol{x}) N_B(\boldsymbol{x}) \mathrm{d}\boldsymbol{x} $\\

y a  nivel de elemento

$K_{ab} = a(N_a(\boldsymbol{x}), N_b(\boldsymbol{x})) = \int_{\Omega_e} gh N_a,_i(\boldsymbol{x}) N_b,_i(\boldsymbol{x}) \mathrm{d}\boldsymbol{x} $\\

$M_{ab} = (N_a(\boldsymbol{x}), N_b(\boldsymbol{x})) = \int_{\Omega_e} N_a(\boldsymbol{x}) N_b(\boldsymbol{x}) \mathrm{d}\boldsymbol{x} $\\

Los elementos utilizados en este estudio fueron elementos isoparam\'etricos Tri3, y las matrices se construyeron sumando las contribuciones de cada elemento, usando los algoritmos de ensamblaje propuestos por Hughes \cite{hughes2000}.






