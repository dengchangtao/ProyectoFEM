 \subsection{Bah\'ia rectangular de ancho unitario}
  Para validar los resultados de la implementaci\'on del algoritmo, y estudiar la convergencia de \'este, se ha escogido una soluci\'on anl\'itica al problema presentado en la ecuaci\'on \eqref{helmholtz}. Este caso corresponde a una bah\'ia cuadrada de largo unitario cuyo interior es $\Omega = [0,1]\times[0,1]$, la cual se encuentra cerrada por bordes impermeables, es decir, $\partial \Omega_g=0$, y fondo a profundidd $h_0\in\mathbb{R^+}$. Por medio de separaci\'on de variables, al asumir que $u$ puede escribirse como $u(x,y)=f(x)g(y)$, y al considerar las condiciones de borde, se deduce que los modos de oscilaci\'on vienen dados por 
  
  \begin{equation}
    \begin{array}{cc}
    u_{nm}=A_{nm}\cos(n\pi)\cos(m\pi) & \text{ con } n,m \in \mathbb{N}_0 \text{ y } A_{nm}\in\mathbb{R}
    \end{array}
    \label{eq:bahia_cerrada_modo}
  \end{equation}

  y el per\'iodo de oscilaci\'on asociado
  
  \begin{equation}
    \begin{array}{cc}
    T_{nm}=\dfrac{2}{\sqrt{gh}}\left( n^2+m^2\right)^{-1/2} & \text{ con } n,m \in \mathbb{N}_0
    \end{array}
    \label{eq:bahia_cerrada_periodo}
  \end{equation}
